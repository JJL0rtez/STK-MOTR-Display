\documentclass[a4paper,12pt]{article}

% Packages for styling
\usepackage{graphicx}  % For images (if needed)
\usepackage{array}     % For better table formatting
\usepackage{xcolor}    % For text highlighting
\usepackage{longtable} % For longer tables

% Title & Author
\title{User Manual: Editable Table Guide}
\author{Your Name}
\date{\today}

\begin{document}

\maketitle

\section{How to Use the Editable Table with Formatting and Class Changes}

This section explains how to interact with the editable table, modify its contents, and understand the formatting rules.

\subsection{Basic Actions}
\begin{itemize}
    \item \textbf{Click a cell} to edit its text.
    \item \textbf{Press Enter or click outside} to save changes.
    \item \textbf{If the text starts with "@",} the cell class changes to \texttt{event-over} (gray background).
    \item \textbf{If "moved ring \#\#" appears,} it is highlighted in \textcolor{red}{red and bold}.
    \item \textbf{If a cell with "event-over" does not contain "@",} it resets back to \texttt{event} (blue background).
\end{itemize}

\subsection{Example Use Cases}
\begin{longtable}{|p{4cm}|p{4cm}|p{6cm}|}
\hline
\textbf{Input Text} & \textbf{Class Change} & \textbf{Formatted Output} \\ \hline
\texttt{@Meeting 10AM} & \texttt{event} $\rightarrow$ \texttt{event-over} & \texttt{Meeting 10AM} (Gray background) \\ \hline
\texttt{Sparring - moved ring 2} & No class change & \texttt{Sparring -} \textbf{\textcolor{red}{moved ring 2}} \\ \hline
\texttt{moved ring 5} & No class change & \textbf{\textcolor{red}{moved ring 5}} \\ \hline
\texttt{Regular Event} (was \texttt{event-over}) & \texttt{event-over} $\rightarrow$ \texttt{event} & \texttt{Regular Event} (Blue background) \\ \hline
\end{longtable}

\subsection{Step-by-Step Guide}
\begin{enumerate}
    \item \textbf{Click on any table cell} to enable editing.
    \item \textbf{Type or edit the text.}
    \item \textbf{Press Enter or click outside} to save changes.
    \item \textbf{If the text starts with "@",} the "@" is removed, and the cell background turns gray (\texttt{event-over}).
    \item \textbf{If "moved ring \#\#" appears,} it will be displayed in \textcolor{red}{bold red} within the text.
    \item \textbf{If the "@" is removed,} the cell resets back to the default event format (blue background).
\end{enumerate}

\subsection{Notes}
\begin{itemize}
    \item Clicking a cell enables text editing.
    \item Pressing Enter or clicking outside saves changes.
    \item \textbf{Blue cells} (\texttt{event}) are standard event entries.
    \item \textbf{Gray cells} (\texttt{event-over}) indicate events that started with "@".
    \item \textbf{Red text} highlights any mention of \texttt{"moved ring \#\#"}.
\end{itemize}

\end{document}
